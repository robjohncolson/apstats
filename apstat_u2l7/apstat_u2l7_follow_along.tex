\documentclass[11pt]{article}
\usepackage[margin=1in]{geometry}
\usepackage{amsmath}
\usepackage{amssymb}
\usepackage{enumitem}
\usepackage{graphicx}
\usepackage{tikz}
\usepackage{fancyhdr}
\pagestyle{fancy}
\lhead{AP Statistics - Topic 2.7}
\chead{Residuals Follow-Along}
\rhead{Name: \underline{\hspace{2in}}}

\begin{document}

\section*{Video 1: Calculating and Interpreting Residuals}

\subsection*{Part A: Building Understanding (0:00 - 1:37)}

\begin{enumerate}[label=\arabic*.]
\item \textbf{Context Connection:} The study examines supermarkets in San Antonio, Texas. Before watching, predict: Would you expect stores in wealthier neighborhoods to have more or fewer organic food options? Why?

\vspace{0.5in}

\item \textbf{Data Interpretation (1:13-1:37):} As you view the scatter plot:
\begin{enumerate}[label=(\alph*)]
\item The regression equation is $\hat{y} = 0.0007x + 2.76$. What does the slope of 0.0007 tell us in context?

\vspace{0.75in}

\item For every \$10,000 increase in average household income, how many additional organic items would we expect?

\vspace{0.5in}
\end{enumerate}
\end{enumerate}

\subsection*{Part B: Understanding Residuals (1:37 - 3:35)}

\begin{enumerate}[label=\arabic*., resume]
\item \textbf{Definition Focus (1:37-1:57):} Complete the formula:
$$\text{Residual} = \underline{\hspace{1in}} - \underline{\hspace{1in}} = y - \hat{y}$$

\item \textbf{Calculation Practice (2:28-3:23):} Work along with the example:
\begin{enumerate}[label=(\alph*)]
\item Given: Income = \$66,703, Actual organic items = 84
\item Show the prediction calculation:
$$\hat{y} = 0.0007(\underline{\hspace{0.75in}}) + 2.76 = \underline{\hspace{0.5in}}$$
\item Calculate the residual:
$$\text{Residual} = \underline{\hspace{0.3in}} - \underline{\hspace{0.3in}} = \underline{\hspace{0.3in}}$$
\end{enumerate}

\item \textbf{Interpretation (3:23-3:53):} Complete this interpretation table:
\begin{center}
\begin{tabular}{|l|l|l|}
\hline
\textbf{Residual Sign} & \textbf{Meaning} & \textbf{Actual vs. Predicted} \\
\hline
Positive (+) & Model \underline{\hspace{1.5in}} & Actual \underline{\hspace{0.3in}} Predicted \\
\hline
Negative (-) & Model \underline{\hspace{1.5in}} & Actual \underline{\hspace{0.3in}} Predicted \\
\hline
\end{tabular}
\end{center}

\item \textbf{Critical Thinking:} A store has a residual of -15.2 organic items. Write a complete sentence interpreting this in context.

\vspace{0.75in}
\end{enumerate}

\subsection*{Part C: Constructing Residual Plots (3:53 - 5:22)}

\begin{enumerate}[label=\arabic*., resume]
\item \textbf{Axes Understanding (4:03-4:42):} In a residual plot:
\begin{enumerate}[label=(\alph*)]
\item What variable is on the x-axis? \underline{\hspace{2in}}
\item What is plotted on the y-axis? \underline{\hspace{2in}}
\item Where is the reference line drawn? At y = \underline{\hspace{0.3in}}
\end{enumerate}

\item \textbf{Plot Analysis:} Sketch what you expect to see in the residual plot space below. Mark the horizontal reference line clearly.

\begin{center}
\begin{tikzpicture}[scale=0.8]
\draw[->] (0,0) -- (8,0) node[right] {Income};
\draw[->] (0,-2) -- (0,2) node[above] {Residual};
\draw[dashed, gray] (0,0) -- (8,0);
\foreach \y in {-1,1} {
    \draw (0,\y) -- (-0.1,\y) node[left] {\y};
}
\end{tikzpicture}
\end{center}

\item \textbf{Prediction (5:16-5:22):} Before watching Video 2, what might it mean if all the residuals were above the zero line? What about if they showed a curved pattern?

\vspace{0.75in}
\end{enumerate}

\newpage

\section*{Video 2: Assessing Model Fit with Residual Plots}

\subsection*{Part D: Recognizing Good Fit (1:26 - 2:23)}

\begin{enumerate}[label=\arabic*., resume]
\item \textbf{Pattern Recognition:} When a linear model is a good fit, the residual plot shows:
\begin{itemize}
\item Apparent \underline{\hspace{1in}} in the residual plot
\item Scatter centered at \underline{\hspace{0.5in}}
\item No clear \underline{\hspace{1in}}
\end{itemize}

\item \textbf{Conceptual Understanding:} Why does ``random noise'' in a residual plot indicate a good model? (Hint: What has the model captured vs. what remains?)

\vspace{1in}
\end{enumerate}

\subsection*{Part E: Recognizing Poor Fit (2:23 - 3:26)}

\begin{enumerate}[label=\arabic*., resume]
\item \textbf{Pattern Detection (2:39-3:05):} Describe the pattern seen in the ``bad fit'' example:

\vspace{0.75in}

\item \textbf{Diagnostic Skills:} Match each residual plot description with what it suggests:
\begin{center}
\begin{tabular}{|l|l|}
\hline
\textbf{Residual Plot Shows} & \textbf{Suggests} \\
\hline
Random scatter around zero & A. Curvature in the relationship \\
\hline
U-shaped or inverted U pattern & B. Linear model is appropriate \\
\hline
Fan shape (spreading out) & C. Changing variability \\
\hline
\end{tabular}
\end{center}
\end{enumerate}

\subsection*{Part F: Application and Analysis (3:26 - 4:31)}

\begin{enumerate}[label=\arabic*., resume]
\item \textbf{Real Data Analysis (3:26-3:57):} Looking at the grocery store residual plot:
\begin{enumerate}[label=(\alph*)]
\item What evidence suggests the model might be adequate?

\vspace{0.75in}

\item What subtle patterns raise concerns about the fit?

\vspace{0.75in}
\end{enumerate}

\item \textbf{Decision Making:} You've created a linear model and the residual plot shows a clear curved pattern. What are your options? List at least two approaches:

\begin{enumerate}[label=\roman*.]
\item \underline{\hspace{3in}}
\item \underline{\hspace{3in}}
\end{enumerate}
\end{enumerate}

\subsection*{Part G: Synthesis and Reflection}

\begin{enumerate}[label=\arabic*., resume]
\item \textbf{Essential Knowledge Check:} Write the complete process for assessing a linear model:
\begin{enumerate}[label=Step \arabic*:]
\item Fit the linear model and find $\hat{y} = $ \underline{\hspace{1.5in}}
\item Calculate residuals using: \underline{\hspace{2in}}
\item Create a residual plot with \underline{\hspace{1in}} on x-axis and \underline{\hspace{1in}} on y-axis
\item Look for \underline{\hspace{1in}} to confirm good fit, or \underline{\hspace{1in}} to identify poor fit
\end{enumerate}

\item \textbf{Real-World Connection:} Why might it matter ethically if grocery stores in lower-income areas have fewer healthy food options? How could this analysis help address the issue?

\vspace{1in}

\item \textbf{Extension Thinking:} If you were advising a grocery chain, what other variables besides income might predict organic food availability? List two and explain why:

\vspace{1in}
\end{enumerate}

\section*{Quick Check (Complete after both videos)}

\begin{enumerate}[label=\Roman*.]
\item A residual of zero means:
\begin{enumerate}[label=\Alph*.]
\item The model is perfect
\item The predicted value equals the actual value for that point
\item The linear model is inappropriate
\item The point is an outlier
\end{enumerate}

\item Random scatter in a residual plot is:
\begin{enumerate}[label=\Alph*.]
\item Evidence the model is wrong
\item Evidence of measurement error
\item Evidence the linear model is appropriate
\item Evidence we need more data
\end{enumerate}

\item If residuals show a pattern, we should:
\begin{enumerate}[label=\Alph*.]
\item Ignore it and use the model anyway
\item Consider a different type of model
\item Collect more data
\item Remove the outliers
\end{enumerate}
\end{enumerate}

\vspace{0.5in}
\noindent\textbf{Key Formulas to Remember:}
\begin{itemize}
\item Residual = Actual - Predicted = $y - \hat{y}$
\item Positive residual $\rightarrow$ Model underestimated
\item Negative residual $\rightarrow$ Model overestimated
\item Random residuals $\rightarrow$ Good linear fit
\item Patterned residuals $\rightarrow$ Consider non-linear model
\end{itemize}

\end{document}