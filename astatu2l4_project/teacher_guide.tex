\documentclass[11pt]{article}
\usepackage{geometry}
\usepackage{amsmath}
\usepackage{amssymb}
\usepackage{enumitem}
\usepackage{graphicx}
\usepackage{fancyhdr}
\usepackage{tcolorbox}
\usepackage{array}
\usepackage{longtable}
\usepackage{booktabs}

\geometry{margin=1in}
\pagestyle{fancy}
\fancyhf{}
\lhead{AP Statistics - Unit 2}
\rhead{Teacher Implementation Guide}
\cfoot{\thepage}

\begin{document}

\begin{center}
\LARGE \textbf{Teacher Guide for Scatterplot Correlation Project}
\end{center}

\section{Purpose and Learning Objectives}
This hands-on project strengthens student understanding of bivariate relationships, specifically:
\begin{itemize}
    \item Describing association (form, direction, strength) in context
    \item Quantifying linear strength using correlation coefficient $r$
    \item Applying the ellipse method as a visual tool for estimating $r$
    \item Distinguishing correlation from causation
\end{itemize}

\section{AP Statistics Standards Alignment}

\begin{longtable}{|p{2.5cm}|p{9cm}|p{3cm}|}
\hline
\textbf{Standard} & \textbf{Description} & \textbf{Project Component} \\
\hline
\endfirsthead
\hline
\textbf{Standard} & \textbf{Description} & \textbf{Project Component} \\
\hline
\endhead
DAT-1.A.1 & Describe patterns in bivariate data including form, direction, strength, and unusual features & DUFS statement \\
\hline
DAT-1.A.2/3 & Identify and explain positive vs negative associations in context & Direction identification \\
\hline
DAT-1.A.5 & Describe strength of association using appropriate language & Strength description with $r$ \\
\hline
DAT-1.B.1 & Calculate and interpret correlation coefficient $r$ & Ellipse method calculation \\
\hline
DAT-1.C.1 & State properties of $r$ (unit-free, $-1 \leq r \leq 1$) & Understanding $r$ bounds \\
\hline
DAT-1.C.2 & Explain that correlation does not imply causation & Causation disclaimer \\
\hline
DAT-1.A.3 & Identify explanatory and response variables & Variable labeling \\
\hline
\end{longtable}

\section{Pre-Class Preparation (15 minutes)}

\subsection{Materials to Prepare}
\begin{itemize}
    \item \textbf{Google Form} with 6--8 quick survey questions (see suggestions below)
    \item \textbf{Equal-scale graph paper templates} (print on 11×17 paper if possible)
    \item \textbf{Station supplies} for measurement pairs:
    \begin{itemize}
        \item Tape measures/meter sticks (at least 3)
        \item Rulers for reaction test (2--3)
        \item Stopwatches or phone timers
        \item Masking tape for marking distances
    \end{itemize}
    \item \textbf{Sticky notes} in two colors for gallery walk
    \item \textbf{Posted materials:}
    \begin{itemize}
        \item Variable choice board (on board or projected)
        \item DUFS sentence frames
        \item Equal scaling reminder poster
    \end{itemize}
\end{itemize}

\subsection{Google Form Survey Questions}
Include these for shared data collection:
\begin{enumerate}
    \item Hours of sleep last night (to nearest 0.5)
    \item Screen time yesterday (hours)
    \item Minutes since you last ate
    \item Current hunger level (0--10 scale)
    \item Number of apps on your phone
    \item Birthday (day of year, 1--365)
    \item Number of siblings
    \item Estimated hours spent on homework weekly
\end{enumerate}

\section{Class Timeline (45 minutes total)}

\begin{tcolorbox}[colback=green!5!white,colframe=green!50!black]
\textbf{Timing Adjustment:} Full activity requires 45 minutes (40 min work + 5 min gallery)
\end{tcolorbox}

\subsection{Launch (0--5 minutes)}
\begin{itemize}
    \item Show example scatterplot on board
    \item Quick review: ``What's the direction? Strength?''
    \item Emphasize: \textbf{``Equal scales are essential for the ellipse method!''}
    \item Display the ellipse formula prominently
\end{itemize}

\subsection{Group Formation \& Variable Selection (5--8 minutes)}
\begin{itemize}
    \item Form groups of 2--3 students
    \item Each group selects unique variable pair from board
    \item Assign roles: data collector, plotter, calculator, writer
    \item Distribute materials to each group
\end{itemize}

\subsection{Data Collection Phase (8--20 minutes)}
\begin{itemize}
    \item \textbf{All students:} Complete Google Form (2 min)
    \item \textbf{Survey variable groups:} Pull data from form responses
    \item \textbf{Measurement groups:} Set up station and collect from 12--18 classmates
    \item \textbf{Teacher:} Circulate to ensure efficient data collection
\end{itemize}

\subsection{Plotting \& Analysis (20--32 minutes)}
\begin{itemize}
    \item \textbf{Minutes 20--24:} Create scatterplot with equal scales
    \item \textbf{Minutes 24--28:} Draw ellipse, mark axes
    \item \textbf{Minutes 28--32:} Calculate $r$, determine sign
    \item \textbf{Teacher:} Check equal scaling before groups draw ellipses!
\end{itemize}

\subsection{DUFS Statement Writing (32--37 minutes)}
\begin{itemize}
    \item Groups write contextual conclusions
    \item Include all required elements
    \item Add ``correlation $\neq$ causation'' statement
\end{itemize}

\subsection{Gallery Walk (37--45 minutes)}
\begin{itemize}
    \item Post all posters
    \item Students rotate with sticky notes
    \item Quick whole-class debrief: Which shows strongest $|r|$? Any surprises?
\end{itemize}

\section{Variable Pair Options}

\begin{center}
\begin{tabular}{|l|l|c|l|}
\hline
\textbf{Variable Pair} & \textbf{Expected Pattern} & \textbf{Expected $r$} & \textbf{Type} \\
\hline
Height vs Wingspan & Strong positive linear & $+0.8$ to $+0.95$ & Measurement \\
Hand span vs Foot length & Moderate positive & $+0.5$ to $+0.7$ & Measurement \\
Shoe size vs Height & Strong positive & $+0.7$ to $+0.9$ & Mixed \\
\hline
Screen time vs Sleep hours & Moderate negative & $-0.3$ to $-0.6$ & Survey \\
Ruler drop vs Reaction taps & Moderate negative & $-0.4$ to $-0.7$ & Measurement \\
\hline
Minutes since ate vs Hunger & Curved positive & $r$ misleading & Survey \\
Phone apps vs Homework hrs & Weak/none & $-0.2$ to $+0.2$ & Survey \\
\hline
Birthday vs Shoe length & No association & $-0.1$ to $+0.1$ & Mixed \\
Birth month vs Siblings & No association & $-0.1$ to $+0.1$ & Survey \\
\hline
\end{tabular}
\end{center}

\section{Assessment Rubric}

\begin{center}
\begin{tabular}{|p{4cm}|p{7cm}|c|}
\hline
\textbf{Component} & \textbf{Criteria} & \textbf{Points} \\
\hline
\textbf{Data \& Labels} & 
\begin{itemize}[noitemsep,topsep=0pt,leftmargin=*]
    \item Variables clearly identified with units
    \item $x$ and $y$ axes properly labeled
    \item Data table complete (12--18 pairs)
\end{itemize} & 2 \\
\hline
\textbf{Graph Quality} & 
\begin{itemize}[noitemsep,topsep=0pt,leftmargin=*]
    \item Equal scales used and explicitly stated
    \item Points plotted accurately
    \item Axes have appropriate range
\end{itemize} & 2 \\
\hline
\textbf{Ellipse \& $r$ Calculation} & 
\begin{itemize}[noitemsep,topsep=0pt,leftmargin=*]
    \item Symmetric ellipse drawn
    \item Major/minor axes marked and measured
    \item Formula correctly applied
    \item Correct sign assigned
\end{itemize} & 3 \\
\hline
\textbf{DUFS Statement} & 
\begin{itemize}[noitemsep,topsep=0pt,leftmargin=*]
    \item Form, direction, strength described
    \item Uses calculated $r$ value
    \item Context maintained throughout
    \item Includes causation disclaimer
\end{itemize} & 3 \\
\hline
\multicolumn{2}{|r|}{\textbf{Total}} & \textbf{10} \\
\hline
\end{tabular}
\end{center}

\section{Common Issues \& Solutions}

\begin{tcolorbox}[colback=red!5!white,colframe=red!50!black,title=\textbf{Critical Issues to Monitor}]
\begin{enumerate}
    \item \textbf{Unequal Scales}
    \begin{itemize}
        \item Check each group's axes before they draw ellipse
        \item Have them write ``1 cm = \_\_ units'' for both axes
        \item Provide pre-scaled templates if needed
    \end{itemize}
    
    \item \textbf{Ellipse Drawing}
    \begin{itemize}
        \item Emphasize symmetry
        \item Major axis goes through ``longest'' direction
        \item Minor axis is perpendicular to major
    \end{itemize}
    
    \item \textbf{Sign Confusion}
    \begin{itemize}
        \item Positive: points trend up-right
        \item Negative: points trend down-right
        \item Near zero: no clear trend
    \end{itemize}
\end{enumerate}
\end{tcolorbox}

\section{Differentiation Strategies}

\subsection{For Struggling Students}
\begin{itemize}
    \item Provide pre-made equal-scale grids
    \item Pair with stronger peer
    \item Give ``high correlation'' pairs (Height vs Wingspan)
    \item Provide completed example as reference
\end{itemize}

\subsection{For Advanced Students}
\begin{itemize}
    \item Have them calculate actual $r$ using technology
    \item Compare ellipse estimate to calculated value
    \item Explore residual plots (preview of Unit 2.6)
    \item Investigate effect of outliers on $r$
\end{itemize}

\section{Extensions for Next Class}

\begin{enumerate}
    \item \textbf{Technology Comparison}
    \begin{itemize}
        \item Calculate actual $r$ using Desmos or calculator
        \item Discuss sources of discrepancy with ellipse method
    \end{itemize}
    
    \item \textbf{Regression Preview} (Unit 2.6)
    \begin{itemize}
        \item Add least-squares regression line
        \item Calculate and interpret slope
        \item Make predictions using the model
    \end{itemize}
    
    \item \textbf{Class Meta-Analysis}
    \begin{itemize}
        \item Combine all groups' $r$ values
        \item Discuss range of correlations observed
        \item Which variables were most predictable?
    \end{itemize}
\end{enumerate}

\section{Safety Reminders}
\begin{itemize}
    \item No collection of weight, grades, income, or medical information
    \item All participation is voluntary with opt-out options
    \item Measurement activities should be done safely (no running for reaction tests)
\end{itemize}

\section{Success Indicators}
Students successfully complete this project when they:
\begin{itemize}
    \item Accurately identify direction from their scatterplot
    \item Reasonably estimate $r$ using the ellipse method
    \item Describe strength using their calculated $r$
    \item Write conclusions in context
    \item Distinguish correlation from causation
\end{itemize}

\vspace{1cm}
\noindent\textbf{Note:} This project directly addresses the learning gaps you identified—students will leave with a tactile, visual understanding of correlation that connects the abstract $r$ value to the concrete pattern they see in their data.

\end{document}