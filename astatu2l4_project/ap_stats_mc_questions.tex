\documentclass[11pt]{article}
\usepackage[margin=1in]{geometry}
\usepackage{enumitem}
\usepackage{amsmath}
\usepackage{tcolorbox}
\usepackage{multicol}

\setlength{\parindent}{0pt}
\setlength{\parskip}{0.5em}

\begin{document}

\begin{center}
{\Large \textbf{AP Statistics - Unit 2, Lesson 4}}\\
{\large \textbf{Conceptual Check: Explanatory vs. Response Variables \& Scatter Plots}}\\
\vspace{0.5em}
Name: \underline{\hspace{3in}} \quad Date: \underline{\hspace{1.5in}}
\end{center}

\hrule
\vspace{1em}

\textbf{Instructions:} Choose the best answer for each question. Each question tests your understanding of the concepts, not memorization of specific examples.

\vspace{0.5em}

% Question 1
\begin{tcolorbox}[colback=blue!5,colframe=blue!40]
\textbf{Question 1:} A researcher notices that students who drink more coffee tend to have higher anxiety levels. When creating a scatter plot to explore this relationship, which statement best explains the variable placement?

\begin{enumerate}[label=\alph*)]
\item Coffee consumption goes on the y-axis because it's the variable being measured in cups, while anxiety goes on the x-axis because it's measured on a scale
\item Anxiety level goes on the y-axis because we suspect it responds to coffee consumption, which goes on the x-axis as the potential explanatory factor
\item Both variables could go on either axis because correlation doesn't imply causation, making the distinction between explanatory and response arbitrary
\item Coffee consumption goes on the y-axis because it's the independent variable that students control, while anxiety is the dependent outcome
\end{enumerate}
\end{tcolorbox}

% Question 2
\begin{tcolorbox}[colback=green!5,colframe=green!40]
\textbf{Question 2:} When examining the relationship between two quantitative variables, a student argues: "Since we found a strong pattern in our scatter plot, the explanatory variable must be causing changes in the response variable." What is the most important statistical principle this student is overlooking?

\begin{enumerate}[label=\alph*)]
\item Strong patterns require correlation coefficients above 0.7 to establish any meaningful relationship between the variables
\item Scatter plots can only show associations between variables, not prove that one variable causes changes in another
\item Explanatory variables must be manipulated in an experiment before any causal claims can be considered
\item The direction of causation might be reversed, with the response variable actually causing the explanatory variable
\end{enumerate}
\end{tcolorbox}

% Question 3
\begin{tcolorbox}[colback=yellow!5,colframe=yellow!40]
\textbf{Question 3:} A school administrator claims: "Since attendance explains test scores, improving attendance will automatically improve test scores." Which statistical concept best explains why this reasoning might be flawed?

\begin{enumerate}[label=\alph*)]
\item Explanatory variables in observational studies only describe patterns, not guaranteed mechanisms for change
\item Test scores might actually be the explanatory variable, with poor performance causing lower attendance
\item The relationship between attendance and test scores is always perfectly linear and predictable
\item Scatter plots cannot be used to analyze relationships between school-related variables
\end{enumerate}
\end{tcolorbox}

% Question 4
\begin{tcolorbox}[colback=purple!5,colframe=purple!40]
\textbf{Question 4:} Two students create scatter plots of the same data but switch which variable goes on each axis. Which statement about their plots is most accurate?

\begin{enumerate}[label=\alph*)]
\item Both plots are equally correct because variable placement is purely a matter of personal preference in data visualization
\item The plot with the explanatory variable on the x-axis better represents the hypothesized relationship, though both show the same association
\item Only one plot is mathematically valid, as the correlation coefficient changes when axes are switched
\item The plots will show completely different patterns, making it impossible to draw the same conclusions
\end{enumerate}
\end{tcolorbox}

% Question 5
\begin{tcolorbox}[colback=red!5,colframe=red!40]
\textbf{Question 5:} When analyzing bivariate quantitative data, researchers designate one variable as "explanatory" and another as "response." What does this designation primarily reflect?

\begin{enumerate}[label=\alph*)]
\item A proven causal mechanism where changes in one variable directly produce changes in the other
\item A hypothesis about which variable might predict or account for variation in the other variable
\item A mathematical requirement for calculating correlation coefficients and regression equations correctly
\item A random choice that has no impact on analysis or interpretation of the data
\end{enumerate}
\end{tcolorbox}

% Question 6
\begin{tcolorbox}[colback=orange!5,colframe=orange!40]
\textbf{Question 6:} A researcher finds that neighborhoods with more parks have lower crime rates. Before concluding that "building parks reduces crime," what is the most important consideration?

\begin{enumerate}[label=\alph*)]
\item Whether the correlation coefficient is positive or negative in the scatter plot
\item Whether confounding variables like income or community investment might explain both phenomena
\item Whether crime rate or number of parks should be the explanatory variable
\item Whether the data includes at least 30 neighborhoods for statistical significance
\end{enumerate}
\end{tcolorbox}

% Question 7
\begin{tcolorbox}[colback=teal!5,colframe=teal!40]
\textbf{Question 7:} A student creates a scatter plot without labels, title, or scale markings but with all points correctly plotted. What is the primary problem with this approach?

\begin{enumerate}[label=\alph*)]
\item The correlation coefficient cannot be calculated without proper labels on the axes
\item Other people cannot verify the context, units, or scale of the relationship being displayed
\item The scatter plot is mathematically invalid without these required components
\item The association between variables will appear weaker without proper formatting
\end{enumerate}
\end{tcolorbox}

% Question 8
\begin{tcolorbox}[colback=gray!5,colframe=gray!40]
\textbf{Question 8:} In the video's example about income and test scores, the presenter emphasizes that the data shows averages for groups, not individuals. Why is this distinction critical?

\begin{enumerate}[label=\alph*)]
\item Individual data points would create a scatter plot while group averages cannot be plotted
\item Group trends don't determine individual outcomes, and individuals may vary greatly from their group's average
\item Averages are always more accurate than individual measurements in statistical analysis
\item Individual performance data is protected by privacy laws and cannot be analyzed statistically
\end{enumerate}
\end{tcolorbox}

% Question 9
\begin{tcolorbox}[colback=cyan!5,colframe=cyan!40]
\textbf{Question 9:} When deciding whether to use a scatter plot for your data, which consideration is most important?

\begin{enumerate}[label=\alph*)]
\item Whether you have exactly two quantitative variables that you want to explore for a possible relationship
\item Whether your hypothesis predicts a positive or negative correlation between the variables
\item Whether you have at least 50 data points to ensure statistical reliability
\item Whether one variable is clearly categorical and the other is clearly quantitative
\end{enumerate}
\end{tcolorbox}

% Question 10
\begin{tcolorbox}[colback=pink!5,colframe=pink!40]
\textbf{Question 10:} A student observes: "Hours studied is on the x-axis and test score is on the y-axis in our scatter plot, so studying must cause higher scores." What would be the best statistical response?

\begin{enumerate}[label=\alph*)]
\item The placement suggests we're exploring if studying predicts scores, but it doesn't prove studying causes improvement
\item The axes should be reversed because test scores actually determine how much students choose to study
\item This conclusion is correct because the explanatory variable always causes the response variable
\item More data points are needed before any relationship between the variables can be suggested
\end{enumerate}
\end{tcolorbox}

\newpage

\section*{Answer Key with Explanations}

\begin{enumerate}
\item \textbf{B} - The explanatory variable (what we think might explain) goes on x-axis, response (what responds) on y-axis

\item \textbf{B} - Scatter plots show associations/correlations but cannot prove causation

\item \textbf{A} - Explanatory variables in observational studies describe patterns but don't guarantee that manipulating them will cause changes

\item \textbf{B} - The choice of axes reflects our hypothesis about the relationship, though the association strength remains the same

\item \textbf{B} - The designation reflects our hypothesis about which variable might predict the other, not proven causation

\item \textbf{B} - Confounding variables (like socioeconomic factors) might explain both variables without one causing the other

\item \textbf{B} - Without context, units, and scale, others cannot properly interpret or verify the displayed relationship

\item \textbf{B} - Group averages don't determine individual outcomes; individuals can vary greatly from their group's trend

\item \textbf{A} - Scatter plots are specifically for exploring relationships between two quantitative variables

\item \textbf{A} - Axis placement indicates what relationship we're exploring but doesn't prove causation
\end{enumerate}

\vspace{1em}
\hrule
\vspace{0.5em}

\textbf{Scoring Guide:}
\begin{itemize}
\item 9-10 correct: Excellent conceptual understanding
\item 7-8 correct: Good grasp of key concepts
\item 5-6 correct: Review the distinction between association and causation
\item Below 5: Re-watch video and focus on why we distinguish explanatory from response variables
\end{itemize}

\end{document}